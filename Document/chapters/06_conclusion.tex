% !TeX root = ../main.tex

\chapter{Conclusion}\label{chapter:Conclusion}

When we set out to do this thesis, we had a clear goal in mind, which was find specific factors that would impact the severity of a vulnerability, in our case found by MACKE which uses composite symbolic execution as its type of analysis. This goal on paper sounds straightforward, but it could not be farther form the truth. As seen in this thesis, we had to do everything using a cascading methodology, since every single step depended highly on the one that came right before it.

During the first few phases we can say that our methodology was appropriate, focusing first on the tools that we would be using, and doing a very thorough analysis regarding which of them could be used in tandem. Through the knowledge gained in this phase, with papers such as \parencite{ognawala} and \parencite{thomasThesis} we were able to grasp an idea of what data could have an impact the severity of vulnerabilities. Furthermore, these papers also gave us the idea of standardizing our severity scores, since they were using thei in-house developed assessment metric, but it was clear to see that unless it is standardized it would not gain any tracktion behind it, therefore we selected CVSS 3.0 as it is the most widely adopted standard in regards to vulnerabily assessment \parencite{cvss3}.

The selection of MACKE as our main tool for this thesis was done before it was begun, and we relied heavily on it, since it would need to find vulnerabilities that were also documented in a reliable database, which we also needed to find.

During the research phase we encountered difficulties, mostly when trying to find a database that met our criteria for what a good database would be -- in regards to the topic of this thesis -- which led us to a very thorough investigation process. By finding NVD as a reliable gouvernmental entity that specializes in this area, we were able to start to find some relation between vulnerabilities

On chapter 3 we discuss how we went about designing and implementing all the tools needed in order to be able to setup the experiments, and of course to confirm whether or not our hypothesis, which was that developing a framework, which uses all of the aforementioned procedures, empowers developers and testers, so that they can find vulnerabilities, assess them and correct them as fast as possible, reducing the investment in both testin, was correct.

When we started our experiments, we expected positive results, but the results exceeded our expectations. If we put everything together we can say that all experiments were successful. MACKE was able to find all vulnerabilities found in the NVD database, which made those vulnerabilities a mere subset of the entire MACKE found vulnerability set. Our learning algorhtm had very good results when doing cross validation. Even though cross validation gave us positive results, the only way to find whether or not our results were accurate or close to what an IT security professional would do in reality, we developed the survey, which also gave us some interesting results.

The feedback received through the survey we conducted was mostly positive, with some remarks that focused mostly on user experience, and quality of life changes, rather than our models failing to assess the proper base scores for vulnerable functions.

All in all, we can say that we were able to accomplish our goal, we have found some important impacting factors, in the form of node attributes which were assigned to functions in a callgraph extracted from LLVM bitcode. Said bitcode was later run by MACKE which found its vulnerabilities, of which we had some CVSS3 scores from the NVD database. When putting together all the scores from all of the programs we were able to come up with a learning algorithm that, since our impact factors were apropriate, allowed us to generate a highly accurate model. Said model was assessed by IT security professionals, which also confirmed the cross validation results.

We have laid a strong foundation in this thesis, and with the work such as \parencite{ognawala} and \parencite{novelty} we are certain better results can be obtained when putting all of these ideas together. With the growth of different types of automatic testing and analysis in the IT indrustry we know that the assessment of vulnerabilities with CVSS will become a standard, and when that time comes, we hope that all the work that has been put into this thesis is of help to further explore the possibilities in this area.